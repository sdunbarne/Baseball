%%% -*-LaTeX-*-
%%% baseball.tex.orig
%%% Prettyprinted by texpretty lex version 0.02 [21-May-2001]
%%% on Wed May  4 05:50:37 2022
%%% for Steve Dunbar (sdunbar@family-desktop)

\documentclass[12pt]{article}

\input{../../../../etc/macros}
%% \input{../../../../etc/mzlatex_macros}
\input{../../../../etc/pdf_macros}

\bibliographystyle{plain}

\begin{document}

\myheader \mytitle

\hr

\sectiontitle{A Markov Chain Model of Baseball}

\hr

\usefirefox

\hr

% \visual{Study Tip}{../../../../CommonInformation/Lessons/studytip.png}
% \section*{Study Tip}

% \hr

\visual{Rating}{../../../../CommonInformation/Lessons/rating.png}
\section*{Rating} %one of
% Everyone: contains no mathematics.
% Student: contains scenes of mild algebra or calculus that may require guidance.
Mathematically Mature:  may contain mathematics beyond calculus with
proofs.  % Mathematicians Only: prolonged scenes of intense rigor.

\hr

\visual{Section Starter Question}{../../../../CommonInformation/Lessons/question_mark.png}
\section*{Section Starter Question}

What is the baseball statistic known as \( \mathit{OPS} \)?  How is this
statistic used?

\hr

\visual{Key Concepts}{../../../../CommonInformation/Lessons/keyconcepts.png}
\section*{Key Concepts}

\begin{enumerate}
    \item
        The goal is to model a half-inning of a baseball game as a
        Markov chain to find the expected number of runs a player will
        score as a means of evaluating and comparing players.
    \item
        The expected number of runs a player will score uses the
        standard partition of the transition probability matrix over the
        transient and absorbing states.
    \item
        A convenient way to build the transition probability matrix is
        to compose it from \( 6 \) simpler matrices with entries either \(
        0 \) or \( \rho_x \), corresponding to the \( 6 \) probabilities
        in the hitter's distribution.
\end{enumerate}

\hr

\visual{Vocabulary}{../../../../CommonInformation/Lessons/vocabulary.png}
\section*{Vocabulary}
\begin{enumerate}
    \item
        A new baseball statistic is how many runs will a team score if
        it uses a player for every plate appearance, called \defn{Markov
        runs}.
    \item
        The \defn{On-base plus slugging} statistic or OPS is the sum of
        a player's slugging average and his on-base average.
\end{enumerate}

\hr

\section*{Notation}
\begin{enumerate}
    \item
        \( p_{i,j} \) -- the probability of moving from one state \( i \)
        to state \( j \)
    \item
        \( \mathit{AB} \) denotes at-bats, \( S \) denotes singles, \( D
        \) denotes doubles, \( T \) denotes triples, \( \mathit{HR} \)
        denotes home runs, \( \mathit{BB} \) denotes walks plus hit
        batters, \( \mathit{AVE} \) denotes batting average \( (S + D +
        T + \mathit{HR})/\mathit{AB} \), \( \mathit{SA} \) denotes
        slugging average \( (1 \cdot S + 2 \cdot D + 3 \cdot T + 4 \cdot
        \mathit{HR})/\mathit{AB} \), a measure of the batting
        productivity of a player, \( \mathit{OB} \) denotes on-base
        average \( (\mathit{BB} + S + D + T + \mathit{HR})/(\mathit{AB}+\mathit
        {BB}) \), and \( \mathit{OPS} = \mathit{OB} + \mathit{SA} \)
        measuring the ability of a player both to get on base and to hit
        for power
    \item
        \( (\rho_O, \rho_B, \rho_1, \rho_2, \rho_3 \rho_4) \) --
        probability distribution where \( \rho_O \) is the probability
        of an out, \( \rho_B \) is the probability of a base on balls,
        and \( \rho_i \) is the probability of an \( i \)-base hit.
    \item
        \( R \) -- the \( 25 \times 1 \) column vector containing the
        expected or average runs scored after each state on one play
        only. The elements of \( R \) are \( R_{1} \), \( R_{2} \),
        \dots, \( R_{25} \)
    \item
        \( E \) -- the expected runs in the rest of the inning after any
        runners and outs state
    \item
        \( Q \), \( Q' \) -- submatrices of the transition probability
        matrix corresponding to transitions among the transients and the
        transients to the stationary state.  \( E' \) corresponding
        expected runs subvector.
    \item
        \( I_{24} \) -- \( 24 \times 24 \) identity matrix.
\end{enumerate}

\visual{Mathematical Ideas}{../../../../CommonInformation/Lessons/mathematicalideas.png}
\section*{Mathematical Ideas}
\subsection*{}

%% Introduction and main question
This section assumes familiarity with the rules and terminology of
American baseball%
\index{baseball}
and some of the statistics associated with baseball players.  The goal
is to model a half-inning of a baseball game as a Markov chain to find
the expected number of runs a player will score as a means of evaluating
and comparing players.

\subsection*{States of the Model} The Markov chain will be the sequence
of situations arising in a half-inning of a baseball game.  In this
Markov chain model of baseball each step will be a plate appearance.  A
plate appearance is similar to an at-bat, but includes walks, hit
batters, and sacrifices that recorded baseball does not regard as an
official at-bat.  In this baseball model, the states are the runners on
base together with a count of the present outs.  Each plate appearance
has eight possibilities for the runners on base:
\begin{itemize}
    \item
        the bases can be empty;
    \item
        there can be one runner on first, second, or third base;
    \item
        there can be two runners on base in three different ways; or
    \item
        the bases can be loaded.
\end{itemize}
The number of outs can be \( 0 \), \( 1 \), or \( 2 \) to make
twenty-four states.  All situations with three outs are the same, and
this situation is the final \( 25 \)th absorbing state.  Once the game
reaches this state, the half-inning ends and the Markov chain is
stopped.  This simplified model takes no account of how many players are
on base or where players are located when the third out is made.  Table~%
\ref{tab:baseball:states} below gives the labeling of the \( 24 \)
possible runners and outs states.  Figure~%
\ref{fig:baseball:states} is a diagram of the states along with a few
sample transitions.  The model tracks runs scored separately, runs are
not part of the states.  Osbourne \emph{et al.}
\cite{osbourne20}, use the same set of states with more descriptive
state labels but ordering them left to right in a row, then row by row.
\begin{table}
    \centering
    \begin{tabular}{cccccccccc}
        \toprule Runners: & None & 1st & 2nd & 3rd & 1\&2 & 1\&3 & 2\&3 & 1,2,\&3 \\ 
        \midrule Outs     &      &     &     &     &      &      &      &         \\ 
        0                 & 1    & 4   & 7   & 10  & 13   & 16   & 19   & 22      \\ 
        1                 & 2    & 5   & 8   & 11  & 14   & 17   & 20   & 23      \\ 
        2                 & 3    & 6   & 9   & 12  & 15   & 18   & 21   & 24      \\ 
        3                 & 25   &     &     &     &      &      &      &         \\ 
        \bottomrule
    \end{tabular}
    \caption{States of the Markov chain model of a half inning of
    baseball}%
    \label{tab:baseball:states}
\end{table}

\begin{figure}
    \centering
\begin{asy}
settings.outformat = "pdf";

size(5inches);

real myfontsize = 12;
real mylineskip = 1.2*myfontsize;
pen mypen = fontsize(myfontsize, mylineskip);
defaultpen(mypen);

real R = 1/5;
path diamond = rotate(45) * unitsquare;
path firstbase = shift((sqrt(2)/2, sqrt(2)/2)) * rotate(45) * scale(R) * box((-1/2,-1/2),(1/2,1/2));
path secbase = shift((0, sqrt(2))) * rotate(45) * scale(R) * box((-1/2,-1/2),(1/2,1/2));
path thirdbase = shift((-sqrt(2)/2, sqrt(2)/2)) * rotate(45) * scale(R) * box((-1/2,-1/2),(1/2,1/2));
path home = scale(R) * ((-1/2,-1/4)--(-1/2,3/4)--(1/2,3/4)--(1/2,-1/4)--(0,-1/2)--cycle);
path field[] = diamond ^^ firstbase ^^ secbase ^^ thirdbase ^^ home;

int state = 1;
for (int i=0; i<=7; ++i) {
  for (int j=0; j<=2; ++j) {
    label(scale(1.0) * format("%i", state), (2*i,-2*j));
    ++state;
  }
}

real fieldscale = 0.75;

// no runners
for (int j=0; j<=2; ++j) {
  draw( shift((0,  -2*j-1/2)) * scale(fieldscale) * field );
  filldraw( shift((0, -2*j-1/2)) * scale(fieldscale) * firstbase, white);
  filldraw( shift((0, -2*j-1/2)) * scale(fieldscale) * secbase, white);
  filldraw( shift((0, -2*j-1/2)) * scale(fieldscale) * thirdbase, white);
  filldraw( shift((0, -2*j-1/2)) * scale(fieldscale) * home, white);
}

// runner on first
for (int j=0; j<=2; ++j) {
  draw( shift((2,  -2*j-1/2)) * scale(fieldscale) * field );
  filldraw( shift((2, -2*j-1/2)) * scale(fieldscale) * firstbase, red);
  filldraw( shift((2, -2*j-1/2)) * scale(fieldscale) * secbase, white);
  filldraw( shift((2, -2*j-1/2)) * scale(fieldscale) * thirdbase, white);
  filldraw( shift((2, -2*j-1/2)) * scale(fieldscale) * home, white);
}

// runner on second
for (int j=0; j<=2; ++j) {
  draw( shift((4,  -2*j-1/2)) * scale(fieldscale) * field );
  filldraw( shift((4, -2*j-1/2)) * scale(fieldscale) * firstbase, white);
  filldraw( shift((4, -2*j-1/2)) * scale(fieldscale) * secbase, red);
  filldraw( shift((4, -2*j-1/2)) * scale(fieldscale) * thirdbase, white);
  filldraw( shift((4, -2*j-1/2)) * scale(fieldscale) * home, white);
}

// runner on third
for (int j=0; j<=2; ++j) {
  draw( shift((6,  -2*j-1/2)) * scale(fieldscale) * field );
  filldraw( shift((6, -2*j-1/2)) * scale(fieldscale) * firstbase, white);
  filldraw( shift((6, -2*j-1/2)) * scale(fieldscale) * secbase, white);
  filldraw( shift((6, -2*j-1/2)) * scale(fieldscale) * thirdbase, red);
  filldraw( shift((6, -2*j-1/2)) * scale(fieldscale) * home, white);
}

// runners on first and second
for (int j=0; j<=2; ++j) {
  draw( shift((8,  -2*j-1/2)) * scale(fieldscale) * field );
  filldraw( shift((8, -2*j-1/2)) * scale(fieldscale) * firstbase, red);
  filldraw( shift((8, -2*j-1/2)) * scale(fieldscale) * secbase, red);
  filldraw( shift((8, -2*j-1/2)) * scale(fieldscale) * thirdbase, white);
  filldraw( shift((8, -2*j-1/2)) * scale(fieldscale) * home, white);
}

// runners on first and third
for (int j=0; j<=2; ++j) {
  draw( shift((10,  -2*j-1/2)) * scale(fieldscale) * field );
  filldraw( shift((10, -2*j-1/2)) * scale(fieldscale) * firstbase, red);
  filldraw( shift((10, -2*j-1/2)) * scale(fieldscale) * secbase, white);
  filldraw( shift((10, -2*j-1/2)) * scale(fieldscale) * thirdbase, red);
  filldraw( shift((10, -2*j-1/2)) * scale(fieldscale) * home, white);
}

// runner on second and third
for (int j=0; j<=2; ++j) {
  draw( shift((12,  -2*j-1/2)) * scale(fieldscale) * field );
  filldraw( shift((12, -2*j-1/2)) * scale(fieldscale) * firstbase, white);
  filldraw( shift((12, -2*j-1/2)) * scale(fieldscale) * secbase, red);
  filldraw( shift((12, -2*j-1/2)) * scale(fieldscale) * secbase, red);
  filldraw( shift((12, -2*j-1/2)) * scale(fieldscale) * home, white);
}

// bases loaded
for (int j=0; j<=2; ++j) {
  draw( shift((14,  -2*j-1/2)) * scale(fieldscale) * field );
  filldraw( shift((14, -2*j-1/2)) * scale(fieldscale) * firstbase, red);
  filldraw( shift((14, -2*j-1/2)) * scale(fieldscale) * secbase, red);
  filldraw( shift((14, -2*j-1/2)) * scale(fieldscale) * thirdbase, red);
  filldraw( shift((14, -2*j-1/2)) * scale(fieldscale) * home, white);
}

// Three outs
label(scale(1.0) * "$25$", (0,-6));
filldraw( shift((0, -2*3-1/2)) * scale(fieldscale) * firstbase, white);
filldraw( shift((0, -2*3-1/2)) * scale(fieldscale) * secbase, white);
filldraw( shift((0, -2*3-1/2)) * scale(fieldscale) * thirdbase, white);
filldraw( shift((0, -2*3-1/2)) * scale(fieldscale) * home, white);
draw( shift((0,  -2*3-1/2)) * scale(fieldscale) * field );


real eps = 1/10;
draw( Label("$\rho_O$"), (0, -1/2)--(0, -3/2+eps), Arrow);
draw( Label("$\rho_4$", align=N), (-1/4, 1/4){NW}..(0,1)..(1/4,1/4){SW}, Arrow);
draw( Label("$\rho_1 + \rho_B$", align=3.5S+0.5E), (1/2+eps, 0)--(3/2-eps, 0), Arrow);
draw( Label("$\rho_2$", align=S, position=Relative(0.70)), (1/2+eps, 0)..(2,1)..(4, 0.7), Arrow);
draw( Label("$\rho_3$", align=N), (1/2-eps, eps)..(2, 1.5)..(11/2-eps, 0), Arrow);

label("Outs", (-2,1));
label("$0$", (-2,  0));
label("$1$", (-2, -2));
label("$2$", (-2, -4));
label("$3$", (-2, -6));
label("Runners on base", (7, 2));
\end{asy}
    \caption{The state diagram for the Markov chain model of a
    half-inning of baseball, with only transitions from state 1
    indicated as an example.}%
    \label{fig:baseball:states}
\end{figure}

\subsection*{Transition Probability Matrix} Denote the probability of
moving from one state \( i \) to state \( j \) by \( p_{i,j} \).  Many
transitions are impossible, such as a transition to a state with fewer
outs, and hence have probabilities equal to zero. Using the numbering
above, \( p_{1,4} \) is the probability of going from no outs and none
on to a runner on first with no outs, which in turn is equal to the
probability of a single plus the probability of a walk.  In real
baseball, there are other possibilities for this transition such as a
wild pitch, but this simplified model ignores them. In certain
situations different actions by the batter produce the same effect.  As
can be seen in Figure~%
\ref{fig:baseball:states}, if no one is on base, then a single and a
walk have the same effect.  Sum the probability of each event to give
the state transition probability.

For this model imagine that the player bats with plate outcomes
distributed randomly according to his statistical profile.  Plate
appearances are independent.  In other words, a player's chance of
getting a specific kind of hit or making a specific kind of out is
independent of how many runners are on and the number of outs.  Based on
this profile, compute the transition probabilities for each pair of
states, obtaining the transition matrix.  The transition probabilities
could come from player statistics from the previous year, the current
year, or the career, but must use the same profile each time.  The
transition probability matrix \( P \), whose entries are \( p_{i,j} \)
governs the evolution of the half inning.  By the properties of matrix
multiplication, \( P^2 \) is the transition matrix for sequences of two
plays or batters, \( P^3 \) is the transition matrix for sequences of
three batters, and so on, for a string of identical batters.  This
assumption is useful for computing \emph{average} performance which is
the goal here, but not for specific game situations.

\subsection*{Assumptions of the Model} Encoding the half inning of the
game in the transition matrix like this requires simplifying
assumptions.
\begin{enumerate}
    \item
        The Markov property assumption means that this model does not
        take into account how it arrived at a particular situation.  For
        example, if there is a runner on first and no outs, the Markov
        chain model is not concerned about whether there was a walk or a
        single.
    \item
        More specifically, the Markov property assumption deliberately
        eliminates the possibility of a \emph{clutch hitter}.  Such a
        player allegedly hits better when the situation matters more,
        but numerical studies indicate that the concept is an illusion.
    \item
        A related question is whether players get ``hot'' or ``cold'',
        the model ignores this possibility.
    \item
        This model ignores situations such as stolen bases or pick-offs
        that could occur during the plate appearance, changing the state
        without action from the player in his plate appearance.
    \item
        The model ignores double plays, sacrifice bunts, and sacrifice
        flies.
    \item
        The model ignores runs scored while incurring the third out.
    \item
        An important simplifying assumption is that the matrix is time
        independent, and each plate appearance is an independent event.
        In real baseball, as pitchers and batters face each other
        multiple times, each adjusts to the other, changing the
        probabilities.  Fatigue may also be a factor as a game
        progresses.
    \item
        Another related assumption is that the batter's outcomes do not
        depend on the pitcher and what the pitcher does is not dependent
        on the hitter.  The probabilities remain constant, regardless of
        the current match-up and previous encounters between pitcher and
        hitter.
    \item
        Already mentioned above is the assumption that a single advances
        a runner by two bases.  For example, in states \( 4 \), \( 5 \),
        or \( 6 \) with runner on first, a single by the hitter moves
        the base runner to third base, leaving the chain in states \( 16
        \), \( 17 \), or \( 18 \) respectively.  Also, assume that a
        double advances each runner two bases, so that in states \( 4 \),
        \( 5 \), or \( 6 \) with runner on first, a double by the hitter
        moves the base runner to third base, leaving the chain in states
        \( 19 \), \( 20 \), or \( 21 \).  In contrast, in~%
        \cite{cover77} Cover and Keilers explicitly make the assumption
        that ``All singles and doubles are assumed to be long.  That is,
        a single advances a baserunner two bases, and a double scores a
        runner from first base.'' On the other hand, Pankin~%
        \cite{pankin} is not explicit about this assumption, but the
        sample calculations of \( R \) imply that more detailed base
        running plays have non-zero probability.  For instance, Pankin
        implies the possibility that bases loaded with one out (state \(
        23 \)) can change to state \( 5 \) or \( 8 \) meaning that a
        runner on first, as well as the two runners on second and third,
        can score on a single or a double.  Likewise Pankin implies that
        a change from state \( 23 \) to \( 23 \) is possible which would
        be a single allowing one run from third, advancing the runners
        on first and second one base.  As another example, Osbourne
        \emph{et al.}
        \cite{osbourne20} show how to estimate the probabilities that
        with no outs and a baserunner on first, after a single the
        baserunner advances to second or third, or even scores all the
        way from first. The simpler Markov chain model here does not
        assume these additional baserunning and scoring refinements.
    \item
        Bukiet \emph{et al.}
        \cite{bukiet97} present an algorithm to carry Markov
        calculations for one inning forward across a full game, keeping
        track of the probability distribution of runs scored, instead of
        just a half-inning considered here.
\end{enumerate}
Each assumption can be relaxed at the cost of more states and a larger
transition probability matrix, or using more detailed player statistics
or both.  For example, Osbourne \emph{et al.}
\cite{osbourne20} show how to estimate the probability of a sacrifice
fly or grounding into a double play.  However they specifically neglect
events such as a four-base error in transition \( P_{1,1} \), or
reaching base by error, or by hit-by-pitch, also ignored here.  The
intent here is to show how to model baseball as a relatively simple
Markov chain, in the process creating a new player evaluation statistic.

\subsection*{Player Statistics} A particular batter H determines a
twenty-five by twenty-five matrix \( M_{H} \) from his statistical
profile.  For convenience, write the player's profile as follows,
deviating slightly from the standard listing.  Here \( \mathit{AB} \)
denotes at-bats, \( S \) denotes singles, \( D \) denotes doubles, \( T \)
denotes triples, \( \mathit{HR} \) denotes home runs, \( \mathit{BB} \)
denotes walks plus hit batters, \( \mathit{AVE} \) denotes batting
average \( (S + D + T + \mathit{HR})/\mathit{AB} \), \( \mathit{SA} \)
denotes slugging average \( (1 \cdot S + 2 \cdot D + 3 \cdot T + 4 \cdot
\mathit{HR})/\mathit{AB} \), a measure of the batting productivity of a
player, \( \mathit{OB} \) denotes on-base average \( (\mathit{BB} + S +
D + T + \mathit{HR})/(\mathit{AB}+\mathit{BB}) \), and \( \mathit{OPS} =
\mathit{OB} + \mathit{SA} \) measuring the ability of a player both to
get on base and to hit for power, two important offensive skills.  Note
that the last four elements depend on the first six, and hence are not
needed.  For this model, the number of plate appearances is \( \mathit{AB}
+ \mathit{BB} \), ignoring sacrifice flies as in the official
score-keeping.  In this simplified scenario, all outs are equivalent,
there is no practical difference between a strikeout and a pop-out.  As
a result, values of \( \mathit{OB} \) and \( \mathit{OPS} \) here differ
slightly from the official records.  These statistics give the complete
profile of H for the model.  In other words, the player statistics
determine the probabilities that H makes each kind of hit or out, draws
a walk, gets hit by a pitch, and so on.

%% Example profiles
For example, Table~%
\ref{tab:baseball:playerstats} gives a fictional line for a hypothetical star
player X. Table~%
\ref{tab:baseball:playerstats} also includes 2019 season profiles for
Mike Trout (Los Angeles Angels, American League MVP for 2019) and Cody
Bellinger (Los Angeles Dodgers, National League MVP for 2019).
\begin{table}
    \centering
    \begin{tabular}{lcccccccccc}
        \toprule          & \(\mathit{AB}\) & \(S\) & \(D\) & \(T\) & \(\mathit{HR}\) & \(\mathit{BB}\) & \(\mathit{AVE}\) & \(SA\)  & \(OB\) & \(OPS\) \\ 
        \midrule X        & 500             & 100   & 25    & 5     & 30              & 100             & .320             & .570    & .433   & 1.003   \\ 
        MT                & 470             & 63    & 27    & 2     & 45              & 110             & ,291             & .438    & .645   & 1.083   \\ 
        CB                & 558             & 86    & 34    & 3     & 47              & 95              & .305             & .629    & .406   & 1.035   \\ 
        \bottomrule
    \end{tabular}
    \caption{Statistics for representative players.}%
    \label{tab:baseball:playerstats}
\end{table}
Player X has \( 500 \) at-bats but \( 600 \) plate appearances.  In a
given plate appearance player X has a \( \frac{1}{6} \) chance to hit a
single, a \( \frac{1}{6} \) chance to reach first base by a walk or hit
batter, a \( \frac{1}{120} \) chance to hit a triple, and so on.  In
this way, from the \( 6 \) basic player statistics, compute a
probability distribution \( (\rho_O, \rho_B, \rho_1, \rho_2, \rho_3 \rho_4)
\), where \( \rho_O \) is the probability of an out, \( \rho_B \) is the
probability of a base on balls, and \( \rho_i \) is the probability of
an \( i \)-base hit.

\subsection*{Constructing the Transition Probability Matrix} A state can
change in several ways, for instance state \( 16 \) with runners on
first and third with no outs can change into
\begin{itemize}
    \item
        states \( 17 \) with an out,
    \item
        \( 22 \) with a single advancing the runner on first,
    \item
        \( 16 \) back to itself while scoring a run with a single,
    \item
        \( 19 \) with a double scoring both base runners,
    \item
        \( 10 \) with a triple, or
    \item
        \( 1 \) with a home run.
\end{itemize}
A convenient way to build the transition probability matrix is to
compose it from \( 6 \) simpler matrices with entries either \( 0 \) or \(
\rho_x \), corresponding to the \( 6 \) probabilities in the hitter's
distribution.  For example, for \( \rho_O \), the probability of an out,
the number of outs increases by one and the base runners remain the
same, which makes a regularly shaped transition matrix \( P_0 \), where
states with an index divisible by \( 3 \) transition to the absorbing
state \( 25 \) of three outs.  All other states advance by \( 1 \) index
with probability \( \rho_O \).  Taking \( P_3 \) as another example, a
triple scores all runners already on base, leaves the hitter on third
base and the numbers of outs stays the same. Thus in \( P_3 \) each
state transitions to state \( 10 \), \( 11 \) or \( 12 \) according to
the existing number of outs.  The other \( 4 \) matrices are similar,
but the advance of states is slightly more complicated because of the
assumption that runners on base advance by \( 2 \) bases on a hit.
Finally, the full probability transition matrix is the sum of the \( 6 \)
elementary transition matrices:  \( P = P_O + P_B + P_1 + P_2 + P_3 + P_4
\).  Set \( P_{25, 25} = 1 \) to account for the absorbing end state.
The transition probability matrices are too large to display here, but
the scripts below build the matrices given the values for \( (\rho_O,
\rho_B, \rho_1, \rho_2, \rho_3 \rho_4) \).

%% Markov runs
Keeping track of runs scored will be an important outcome of the model.
The new statistic called Markov runs%
\index{Markov runs}%
is how many runs a team could score by using this player for every plate
appearance.  The complete definition follows. Cover and Keilers
\cite{cover77} call this the \emph{Offensive Earned-Run Average} or OREA
but that term seems to be no longer used.  This statistic, normalized
using nine innings, for all players each day of the season is available
at \link{https://www.usatoday.com/sports/mlb/sagarin/}{USA Today,
Sagarin ratings} labeled as ``markov RPG'' (Runs Per Game). Given the
player's statistical profile, the player bats according to the Markov
chain model until making three outs.  In principle the test is run
thousands of times, and from it the model determines the average number
of runs scored per nine innings.  However, simple results from Markov
chain theory calculate the statistics without needing thousands of
simulations.

\subsection*{Calculating Markov Runs} Let \( R \) be the \( 25 \times 1 \)
column vector containing the expected or average runs scored after each
state on one play only. Denoting the elements of \( R \) by \( R_{1} \),
\( R_{2} \), \dots, \( R_{25} \) some example calculations are:
\begin{align*}
    R_{1}       &= p_{1,1} (\text{ based on a home run}) \\
    R_{2}       &= p_{2,2} (\text{ based on a home run}) \\
    R_{6}       &= p_{6,12} + 2p_{6,3} (\text{ based on triple or a home
    run}) \\
    R_{17}      &= p_{17,17} + p_{17,20} + 2p_{17,11} + 3p_{17,1} \\
    R_{23}      &= 2p_{23,17} + 2p_{23,20} + 3p_{23,11} + 4p_{23,2} \\
    R_{25}      &= 0.
\end{align*}
The vector of runs calculated here will be componentwise slightly less
than the run vector for Cover and Keilers because advances on singles
and doubles are less than they assume.  Likewise, the vector of runs
calculated here will be componentwise less than the run vector for
Pankin because some state transitions he considers are ignored here.

The key output of the Markov chain baseball model is the computation of
the expected runs in the rest of the inning after any runners and outs
state.  Let \( E \) be the \( 25 \times 1 \) column vector containing
these values.  Then,
\[
    E = R + P R + P^2 R + P^3 R + \cdots\,.
\] This equation says that the expected runs after any state is the sum
of the expected runs after one plate appearance, the expected runs after
two plate appearances, and so on theoretically forever.  Of course,
almost surely (in the probability sense!) the half-inning will end
sometime in the \( 25 \)th absorption state of three outs.  Because the
vector of runs calculated here is less than the other run vectors, the
expected run vector, or Markov runs will be slightly less.  Since the \(
25 \)th state with \( 3 \) outs is the only absorbing state and the
other \( 24 \) states are transient, the \( 25 \times 25 \) probability
transition matrix partitions into the block matrix
\[
    P =
    \begin{pmatrix}
        Q       & Q' \\
        0       & 1
    \end{pmatrix}
\] where \( Q \) is \( 24 \times 24 \), and \( Q' \) is \( 24 \times 1 \).
(This block matrix has a different ordering than the canonical form used
in the section on Waiting Times because here it is more natural to order
the absorbing state as the last state \( 25 \).) Since no runs are
scored from the \( 25 \)th absorbing state, just consider
\[
    E' = R' + Q R' + Q^2 R' + Q^3 R' + \cdots\,
\] where \( R' \) and \( E' \) are the \( 24 \times 1 \) subvectors of \(
E \) and \( R \) corresponding to the first 24 entries.  By the standard
matrix theory, this is equal to
\[
    E' = ( I_{24} - Q )^{-1} R'.
\]

The expected runs vector \( E' \) has useful interpretations.  For
example, if \( P \) uses statistics for all events for a league, then \(
E' \) contains the league average expected runs scored in the rest of
the inning from each runners-and-outs state.  With enough statistics,
similar values for teams, a team's home and road games, or for an
individual player can be calculated.  The last case is the per inning
estimate of run scoring if that player batted all the time.  Also,
restricting the transitions to those not influenced by strategies such
as stolen bases or sacrifice bunts allows comparison of expected run
values with games using those strategies.

Note that \( E'_1 \) is the expected runs after the no runners, no out
state in which all innings begin.  Thus \defn{Markov runs per game} \(
9E'_1 \) is the expected number of runs per \( 9 \) half-innings, or per
game. This computation is especially interesting when applied to an
individual player's statistics.  It provides another way of estimating
how many runs per game a player would score by batting all the time.  As
noted above, some baseball statisticians report this number for players,
but much more common is the \defn{On-base plus slugging} statistic or
OPS\@. For OPS, simply add the player's slugging average to his on-base
average.  This simply computed player statistic empirically correlates
fairly well with the expected number of runs scored using the Markov
chain model.  More details are in the article by D'Angelo
\cite{dangelo10}

\subsection*{Changes to the Model} The elementary model here assumes a
series of identical batters, of course this does not happen in real
games.  This has been the major criticism of the applicability of Markov
models.  But emphasizing again, it is possible to have a Markov model
with different transition matrices for each batter, see
\cite{osbourne20} for an example. Then the assumption of stationarity is
dropped and the probability transition matrix changes with each batter
according to that batter's statistics.  Instead of powers of one
transition matrix \( P \), use products of the matrices for each batter:
\( P_1 P_2 \), \( P_1 P_2 P_3 \), etc.  Also, the expected runs on the
next play column vector \( R \) has to be modified for each batter.
With such changes, the expected runs generalizes to
\[
    E = R_{1} + P_1 R_2 + P_1 P_2 R_3 + P_1 P_2 P_3 R_4 + \cdots\,.
\] An additional potential complication is that it may be necessary to
repeat the calculation for several sequences with different first
batters and then weight the results by the probability of specific
batters beginning the sequence.  The model can be expanded in a number
of ways and the utility is limited only by the amount of data available.

\visual{Section Starter Question}{../../../../CommonInformation/Lessons/question_mark.png}
\section*{Section Ending Answer}

OPS is the sum of ``on-base percentage'' and ``slugging percentage'' to
get one number that unites the two.  OPS is meant to combine how well a
hitter can reach base, with how well he can hit for average and for
power.  ``On-base percentage'' refers to how frequently a batter reaches
base per plate appearance.  Times on base include hits, walks and
hit-by-pitches, but do not include errors, times reached on a fielder's
choice or a dropped third strike.  ``Slugging percentage'' represents
the total number of bases a player records per at-bat.

\subsection*{Sources} This section is adapted from the articles
\cite{cover77},
\cite{dangelo10},
\cite{osbourne20} and
\cite{pankin}.  Each of those articles has many further references.  The
online article \link{http://statshacker.com/blog/2018/05/07/the-markov-chain-model-of-baseball/}
{The Markov Chain Model of Baseball} has the same basic information as
this article, but in a condensed form. The definition of OPS is adapted
from \link{https://www.mlb.com/glossary/standard-stats/on-base-plus-slugging}
{On-base Plus Slugging}

\nocite{}
\nocite{}

\hr

\visual{Algorithms, Scripts, Simulations}{../../../../CommonInformation/Lessons/computer.png}
\section*{Algorithms, Scripts, Simulations}

\subsection*{Algorithm}

\subsection*{Algorithm}

\begin{algorithm}[H]
    \DontPrintSemicolon
    \KwData{Basic palyer statistics}
    \KwResult{Markov runs from each state for the player}
    \BlankLine
    \emph{Initialization}\;
    Enter player statistics\;
    Initialize with zero simple matrices for the player statistics\;
    Fill simple matrices with corresponding player statistic\;
    Transition probability matrix is sum of simple matrices\;
    Create runs scored matrix\;
    Expected runs as product of runs scored matrix with player statistics\;
    \BlankLine
    \emph{Calculation of Markov runs}\;
    Direct matrix calculation of Markov runs from submatrix\;
    Load markovchain library and initialize\;
    Calculate Markov runs as expectedRewards()
    \KwRet{Markov runs from each state}
    \caption{Markov chain simulation.}
\end{algorithm}

\subsection*{Scripts}

\begin{lstlisting}

rho0 <- 333/596
rhoB <- 126/596
rho1 <- 63/596
rho2 <- 27/596
rho3 <- 2/596
rho4 <- 45/596

P0 <- matrix(0, nrow=25, ncol=25)
PB <- matrix(0, nrow=25, ncol=25)
P1 <- matrix(0, nrow=25, ncol=25)
P2 <- matrix(0, nrow=25, ncol=25)
P3 <- matrix(0, nrow=25, ncol=25)
P4 <- matrix(0, nrow=25, ncol=25)

##  State          1   2   3   4   5   6   7   8  9   10  11  12  13  14  15  16  17  18  19  20  21  22  23  24
transitionOut <- c(2,  3, 25,  5,  6, 25,  8,  9, 25, 11, 12, 25, 14, 15, 25, 17, 18, 25, 20, 21, 25, 23, 24, 25)
transitionBB  <- c(4,  5,  6, 13, 14, 15, 13, 14, 15, 16, 17, 18, 22, 23, 24, 22, 23, 24, 22, 23, 24, 22, 23, 24)
transition1B  <- c(4,  5,  6, 16, 17, 18,  4,  5,  6,  4,  5,  6, 16, 17, 18, 16, 17, 18, 19, 20, 21, 16, 17, 18)
transition2B  <- c(7,  8,  9, 19, 20, 21,  7,  8,  9,  7,  8,  9, 19, 20, 21, 19, 20, 21,  7,  8,  9, 19, 20, 21)
transition3B  <- c(10, 11, 12, 10, 11, 12, 10, 11, 12, 10, 11, 12, 10, 11, 12, 10, 11, 12, 10, 11, 12, 10, 11, 12)
transitionHR  <- c(1, 2, 3, 1, 2, 3, 1, 2, 3, 1, 2, 3, 1, 2, 3, 1, 2, 3, 1, 2, 3, 1, 2, 3)

for (i in 1:24) {
    P0[i, transitionOut[i] ] <- rho0
    PB[i, transitionBB[i]  ] <- rhoB
    P1[i, transition1B[i]  ] <- rho1
    P2[i, transition2B[i]  ] <- rho2
    P3[i, transition3B[i]  ] <- rho3
    P4[i, transitionHR[i]  ] <- rho4
}

P <- P0 + PB + P1 + P2 + P3 + P4
P[25, 25] = 1

M <- matrix(0, nrow=25, ncol=5)
M[22:24,   1] <- 1
M[7:18,  2:3] <- 1
M[19:24, 2:3] <- 2
M[4:12,    4] <- 1
M[13:21,   4] <- 2
M[22:24,   4] <- 3
M[1:3,     5] <- 1
M[4:12,    5] <- 2
M[13:21,   5] <- 3
M[22:24,   5] <- 4

trout <- c(rhoB, rho1, rho2, rho3, rho4)
runsTrout <- M %*% trout

Q <-  P[1:24, 1:24]
cat( solve(diag(24) - Q, runsTrout[1:24]) )

library("markovchain")
nStates <- 25
baseballStates <- as.character(c(1:nStates))
mcBaseball <- new("markovchain", states = baseballStates, byrow=TRUE, transitionMatrix = P, name="mcTrout")
cat( expectedRewards(mcBaseball, 30, runsTrout) )
\end{lstlisting}

\hr

\visual{Problems to Work}{../../../../CommonInformation/Lessons/solveproblems.png}
\section*{Problems to Work for Understanding}
\renewcommand{\theexerciseseries}{}
\renewcommand{\theexercise}{\arabic{exercise}}

\begin{exercise}
    Steven Strasburg of the Washington Nationals was the 2019 World
    Series MVP\@.  Discuss why this model may not useful for assessing
    Strasburg as a baseball player.
\end{exercise}
\begin{solution}
    In 2019 Strasburg was a pitcher for the Washington Nationals in the
    National League, as he has been since 2010.  As a pitcher, his
    hitting skills and run production are not emphasized, it is his
    pitching that makes him a valuable player.  Furthermore, as a
    pitcher, he appears in fewer games with even fewer plate
    appearances.  That means he has a smaller base of hitting statistics
    for creating the matrix \( P \) so \( P \) may not fully reflect his
    hitting skills.
\end{solution}

\begin{exercise}
    With this model, what is the expected number of at-bats for (\( 9 \)
    copies of) Mike Trout until the third out in a half-inning?
\end{exercise}
\begin{solution}
    Using the R script with Trout's statistics, the expected time to
    absorption starting from the first state of no-outs is \( 5.369 \)
    at-bats per half-inning.
\end{solution}

\begin{exercise}
    Consider a player K who each time at bat either hits a home run with
    probability \( q \) or strikes out with probability \( 1 - q \).
    Write out the non-zero probability transitions in the matrix \( M_{K}
    \).
\end{exercise}
\begin{solution}
    In the matrix \( M_{K} \), all entries are \( 0 \) except for the
    following:
    \begin{itemize}
        \item
            For each \( i \), \( i= 1,\dots, 25 \) and \( j=i \mod 3 \),
            \( (M_K)_{ij} = q \),
        \item
            For each \( i \), \( i= 1,\dots, 25 \) with \( i \ne 0 \mod
            3 \), and \( j=i+1 \) \( (M_K)_{ij} = 1-q \)
        \item
            For each \( i \), \( i= 1,\dots, 25 \) with \( i = 0 \mod 3 \)
            and \( j=25 \) \( (M_K)_{ij} = 1-q \).
    \end{itemize}
    In the special case \( q = 0.1 \), the Markov runs statistics for a
    half-inning is \( 0.333 \).
\end{solution}

\begin{exercise}
    Find the player statistics for the the MVP in the National League
    and American League for the previous complete baseball season and
    compute the Markov runs for each.
\end{exercise}
\begin{solution}
    Depends on the player statistics.
\end{solution}

\begin{exercise}
    Use this model to calculate the Markov runs for Ted Williams and
    Babe Ruth.  Does this settle the debate about who is the greatest
    hitter in baseball history?
\end{exercise}
\begin{solution}
    The lifetime career statistics for Babe Ruth (1914--1935) and Ted
    Williams (1939--1942, 1946--1960) are in Table~%
    \ref{tab:baseball:ruthwilliams}.  (Values derived from \link {https://www.baseball-reference.com/players/}
    {Baseball-Reference}.)
    \begin{table}
        \centering
        \begin{tabular}{lcccccccccc}
            \toprule          & \(\mathit{AB}\) & \(S\) & \(D\) & \(T\) & \(\mathit{HR}\) & \(\mathit{BB}\) & \(\mathit{AVE}\) & \(SA\)  & \(OB\) & \(OPS\) \\ 
            \midrule Ruth     & 8399            & 1517  & 506   & 136   & 714             & 2062            & 0.342            & 0.690   & 0.474  & 1.164   \\ 
            Williams          & 7706            & 1534  & 525   & 71    & 521             & 2021            & 0.344            & 0.634   & 0.482  & 1.116   \\ 
            \bottomrule
        \end{tabular}
        \caption{Statistics for Babe Ruth and Ted Williams.}%
        \label{tab:baseball:ruthwilliams}
    \end{table}
    Then
    \[
        \rho_{BR} = (0.412, 0.246, 0.181, 0.060, 0.016, 0.085)
    \] and
    \[
        \rho_{TW} = (0.394, 0.262, 0.199, 0.068, 0.009, 0.068).
    \] Using these probabilities in the script for Markov runs, the
    value for Babe Ruth is \( 2.83 \) and the value for Ted Williams is \(
    3.05 \). The two values are close but Williams has the slightly
    greater value.

    No statistic can resolve the comparison between these two great
    players.  Each played in different baseball eras, and each had
    particular strengths.  Besides, what is to be the basis for
    comparison, their lifetime statistics used here, or the best career
    year for each?  If the latter, how to choose the best career year?
    Note that Williams lost 3 (possibly best) years of his career to
    military service in World War II\@.  It is interesting to compute
    the Markov runs for each, but the statistic won't answer this
    opinion question.

\end{solution}

\hr

\visual{Books}{../../../../CommonInformation/Lessons/books.png}
\section*{Reading Suggestion:}

\bibliography{../../../../CommonInformation/bibliography}

%   \begin{enumerate}
%     \item
%     \item
%     \item
%   \end{enumerate}

\hr

\visual{Links}{../../../../CommonInformation/Lessons/chainlink.png}
\section*{Outside Readings and Links:}
\begin{enumerate}
    \item
    \item
    \item
    \item
\end{enumerate}

\hr

\section*{\solutionsname} \loadSolutions

\hr

\mydisclaim \myfooter

Last modified:  \flastmod

\end{document}

%%% Local Variables:
%%% mode: latex
%%% TeX-master: t
%%% End:
